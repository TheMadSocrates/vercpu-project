The main objective of this project is to design and implement a simple 8-bit
CPU with memory using the Verilog HDL. In order to learn more about different
CPU implementations, the design uses a horizontal microcode implementation. We
chose horizontal microcode implementation because of the ability to easily
update how instructions are processed and to add more useful instructions for
future revisions. The CPU has 256 bytes of memory with access controlled by a
memory address register and a memory data register. Other than the memory data
register, the CPU has one main register to be able to manipulate data, the
accumulator register. The CPU will recognize and execute 22 different
instructions, outlined below in Appendix \ref{app:proc_iset}, with the exact
microcode implementation represented in the table in
Appendix \ref{app:mcrom}. The arithmetic logic unit, explained in the Datapath
section, will be able to choose from up to 8 operands in conjunction with the
accumulator register. The other objective was to be able to synthesize our
Verilog design into a real FPGA, the Spartan-6 on the Nexys3 board. The 4-digit,
7-segment LED display on the Nexys3 is programmed to display the full 16-bit
instruction for debugging purposes.
